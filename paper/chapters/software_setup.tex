To calculate theoretical bounds and characterise the delivery statistics of Bluetooth-based communication, we used The ONE\footnote{http://akeranen.github.io/the-one/} simulator to simulate the testing environment.
A large 400$\times$400 L-shape building hall was generated in WKT format to be used by the simulator, and 25 static hosts were randomly placed in it.
The distances between all the hosts were evaluated and binned, with a bin width resolution of 5 meters.
A simulator run ends when a specific number of randomly generated messages was created for all bins, and then delivered or discarded.
We wanted to evaluate the performance of the Bluetooth Low Energy running on 1M PHY mode\footnote{\href{Bluetooth SIG}{\url{https://www.bluetooth.com/blog/exploring-bluetooth-5-how-fast-can-it-be/}}}, which means a theoretical bitrate cap of 1Mbps\footnote{\href{Texas Instruments}{\url{https://dev.ti.com/tirex/explore/node?node=A__AQgtSgkM1Yu1Yimn5wFTOw__SIMPLELINK-ACADEMY-CC23XX__gsUPh5j__LATEST}}} at $\le1$ meter, assuming free space propagation (we used the Friis Free Space propagation model \footnote{Cite this? Friis, H. (1946). A note on a simple transmission formula}).
The simulation was repeated for 25 runs per configuration. A configuration is composed of some properties: in this part, we chose to examine the affects of the per-node Bluetooth interface communication range parameter: as wireless signals exponentially degrade with distance, but the received signal strength correlates with the transmitted signal strength, we are interested in the trade-off between energy consumption and the deliverability of messages, as well as the achievable bitrate.
For every simulation run, 1000 messages were generated for every distance bin: the following plot shows a summary of how the message creation distribution varies with range, collected from all runs. We can see a concentration around $\sim$25 meter distance, and a long tail that stretches all the way to $\sim$500 meter.
