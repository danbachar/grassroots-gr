The decentralized administration of the internet is one of the basic enablers of the rapid growth of the Internet as a whole, thanks to the autonomous nature of decision making across different network systems. The core functionality of the Internet is based on a general-purpose solution developed originally by DARPA, uses IP routing using transmission control protocols to make sure a packet travelling along a network path between two distinct hosts will reach its destination reliably.
Establishing, maintaining, and terminating connections between the different Internet autonomous system (ASs) is the task of the border routers between them, which are parts of the infrastructure belonging to the individual Internet Service Providers (ISPs) that together form an Autonomous System (AS).
The ISPs have a central critical role in the functionality of the Internet, as they provide connection to the Internet backhaul to customer end-users. The power dynamic of this system inherently gives control over consumption patterns to the ISPs, as well as monitoring their users behaviour through traffic patterns.
Consolidation of power at the hands of a few ISPs gives them the unique ability to control the flow of information in their AS territory, potentially enabling oppresive practices such as throttling, monitoring, censorship, and surveillance of their consumers. Insofar as authoritative regimes are established over a region, access to content can become severly restricted.
Additionally, an imminent danger to the freedom of speech and freedom of access to information is the increasing trend of media corporations such as Meta and X to employ both explicit and implicit content moderation practices.
Together, the aforementioned practices have in recent years led to a growing interest in decentralized peer-to-peer social networks and messaging platforms. Most notably, the rise of fully-decentralized networks like Mastodon, user-supported decentralized networks such as Bluesky, and end-to-end encrypted decentralized messaging platforms like Briar, Signal, and until recently Firechat, have proven that there is a growing shift in user interest patterns away from traditional Centralized Social Networks (CSNs) towards decentralized social networks of diverse approaches. 
Decentralized social networks can exist either as unstructured or structured peer-to-peer networks, as a democratic ability to run the entire software stack of a server individually (cf. Mastodon), or as a collection of transparent community-operated competing components belonging to the same social network, like in Bluesky.
Structured peer-to-peer networks such as Chord, Kademlia, and the InterPlanetary File System (IPFS) use Distributed Hash Tables (DHTs) to store and retrieve content in a decentralized manner. 
On the other hand, unstructured peer-to-peer networks are more dynamic and do not use DHT to spread centralized information. Instead, unstructured peer-to-peer networks rely on flooding and gossip for content dissemination, like in Gnutella and BitTorrent.
In this research paper, we focus on unstructured peer-to-peer networks, as they are more resilient to network topology changes, peers leaving and joining, and are more suitable for the dynamic nature and high churn of users in disaster-struck areas and refugee camps, as well as keeping more information decentralized and less prone to censorship.