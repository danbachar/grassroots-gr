There is a growing architectural gap between the internet's general-purpose design and the individual implementation of ISPs in their ASes, and the increasing demand for specific functionalities (e.g. Multicast) makes it necessary to implement change when things break \cite{handley}. This is because the Internet was made with the concept of one-size-fits-all, allowing different Internet providers to implement different rules, policies, and techniques to efficitate their networks.
The general-purpose design of the Internet also introduces the immutability of paths between pairs of nodes - once a path between two nodes exists, and the nodes can use it to reach each other, it is a complex process to get the nodes to communicate over new addresses, which happens when those nodes move. In the modern age nodes oftentimes exhibit movement patterns that may render their previous Internet connection point unusable, thereby forcing the nodes to search for a new attachment point - oftentimes switching from WiFi to cellular, between WiFi networks, or cellular to WiFi. In disaster zones, for example after earthquakes, fires, and other natural disasters, existing networking infrastructure is rendered useless due to destruction.
In less-developed areas of the world, such infrastructure might not be available at all: the Laponia region near the Arctic Circle, homeland of the Sámi people, is characterized as a sparsely-populated region full of icy fjords, deep valleys, glaciers and mountains.
The inhabitants of this region have traditionally been reindeer-herders, and often migrate with the reindeer herds along the yearly cycle. Due to the harsh terrain, sparse level of population, and partial autonomy of the Sámi council, the Sámi people lack reliable cellular and internet infrastructure. To provide the Sámi people with modern communication options, the traditional packet-switching paradigm of communication as conceptualized by the Internet is not optimal.
Instead, Lindgren et al. propose the the Store-Carry-Forward paradigm using Delay-Tolerant Networking (DTN) \cite{sami}, where hosts store messages until an opportunity arises to forward them, carry them while moving until they reach a rendezvous with another node or a network attachment point, and then forward them upon contact. This paradigm enables the opportunistic usage of technologies such as Bluetooth to forward messages, which is ubiquitous to virtually all Smartphones and computers today.
Insofar as the Internet is not available, DTNs using Store-Carry-Forward networking can be used as a fallback, being supplemented by traditional Internet infrastruture as it becomes available.
Decentralized social networks can exist either as unstructured or structured peer-to-peer networks, as a democratic ability to run the entire software stack of a server individually (cf. Mastodon), or as a collection of transparent community-operated competing components belonging to the same social network, like in Bluesky.
Structured peer-to-peer networks such as Chord, Kademlia, and the InterPlanetary File System (IPFS) use Distributed Hash Tables (DHTs) to store and retrieve content in a decentralized manner. 
On the other hand, unstructured peer-to-peer networks are more dynamic and do not use DHT to spread centralized information. Instead, unstructured peer-to-peer networks rely on flooding and gossip for content dissemination, like in Gnutella and BitTorrent.
This concept has been the guiding idea of this research paper, with the intent to investigate the reliability and performance of unstructured peer-to-peer networks and grassroots social networks built using the aforementioned paradigm.