Recent events (natural disasters, war, protests) have shown that people heavily rely on online social networks and instant messaging platforms to communicate and coordinate aid efforts during cases such as disaster relief, protests, or emergency response to natural disasters. The existing internet infrastructure is distributed, but its management and ownership is heavily centralized which places power in the hands of a few individuals or companies, that are susceptible to implicit and explicit censorship, while the infrastructure itself is prone to failure in cases of disaster, leaving reliant communities in the dark. In this paper, we propose performance statistics for both simulated as well as real-life unstructured peer-to-peer social networks, as well as investigate and propose resilience metrics for such opportunistic networks.